\documentclass[12pt,a4paper,twoside,openright,titlepage,final]{article}
\usepackage{fontspec}
\usepackage{amsmath}
\usepackage{amsfonts}
\usepackage{amssymb}
\usepackage{makeidx}
\usepackage{graphicx}
\usepackage[hidelinks,unicode=true]{hyperref}
\usepackage[spanish,es-nodecimaldot,es-lcroman,es-tabla,es-noshorthands]{babel}
\usepackage[left=3cm,right=2cm, bottom=4cm]{geometry}
\usepackage{natbib}
\usepackage{microtype}
\usepackage{ifdraft}
\usepackage{verbatim}
\usepackage[obeyDraft]{todonotes}
\ifdraft{
	\usepackage{draftwatermark}
	\SetWatermarkText{BORRADOR}
	\SetWatermarkScale{0.7}
	\SetWatermarkColor{red}
}{}
\usepackage{booktabs}
\usepackage{longtable}
\usepackage{calc}
\usepackage{array}
\usepackage{caption}
\usepackage{subfigure}
\usepackage{footnote}
\usepackage{url}
\usepackage{tikz}
\tikzset{
  treenode/.style = {shape=rectangle, rounded corners,
                     draw, align=center,
                     top color=white, bottom color=blue!20},
  root/.style     = {treenode, font=\Large, bottom color=red!30},
  env/.style      = {treenode, font=\ttfamily\normalsize},
  dummy/.style    = {circle,draw}
}

\setsansfont[Ligatures=TeX]{texgyreadventor}
\setmainfont[Ligatures=TeX]{texgyrepagella}

\input{portada}

\author{José Ignacio Escribano}

\title{Caso práctico II}

\setlength{\parindent}{0pt}

\begin{document}

\pagenumbering{alph}
\setcounter{page}{1}

\portada{Caso Práctico II}{Modelización y tratamiento de la incertidumbre}{Probabilidades y variables aleatorias}{José Ignacio Escribano}{Móstoles}

\listoffigures
\thispagestyle{empty}
\newpage

\tableofcontents
\thispagestyle{empty}
\newpage


\pagenumbering{arabic}
\setcounter{page}{1}

\section{Introducción}


\section{Resolución del caso práctico}



\subsection{Tarea VI}

El beneficio de cada inversión viene dado por la diferencia entre los beneficios netos y la inversión (1 millón de euros). Así, por tanto, si se pierde todo lo invertido tenemos que el beneficio es -1 (los beneficios netos son cero y la inversión 1); cuando no se obtiene beneficio neto ni se pierde la inversión tenemos que los beneficios son 0.\\

Si denotamos con $X_i$ al beneficio neto obtenido en la inversión $i$ con $i = 1,2,3$, tenemos que las funciones de masa vienen dadas de la siguiente forma:

\begin{equation*}
f(X_1 = x) = \begin{cases}
0.1, & \text{si } x = 5 \\
0.3, & \text{si } x = 1 \\
0.6, & \text{si } x = -1
\end{cases}
\end{equation*}

\begin{equation*}
f(X_2 = x) = \begin{cases}
0.2, & \text{si } x = 3 \\
0.4, & \text{si } x = 1 \\
0.4, & \text{si } x = -1
\end{cases}
\end{equation*}

\begin{equation*}
f(X_3 = x) = \begin{cases}
0.1, & \text{si } x = 6 \\
0.7, & \text{si } x = 0 \\
0.2, & \text{si } x = -3
\end{cases}
\end{equation*}

En la Figura~\ref{fig:funciones_masa_inversiones} se pueden ver las funciones de masa de las tres inversiones.\\ 

\begin{figure}[tbph!]
\centering
\includegraphics[width=0.9\linewidth]{imagenes/funciones_masa_inversiones}
\caption{Funciones de masa de las inversiones}
\label{fig:funciones_masa_inversiones}
\end{figure}

Calculamos la media y la varianza del beneficio de cada inversión.

\begin{align*}
E[X_1] & = \sum_{i \in \Omega_1} i\cdot f(X_1 = i) \\ & = 5 \cdot 0.1 + 1 \cdot 0.3 - 1 \cdot 0.6 \\ & = 0.2
\end{align*}

\begin{align*}
Var[X_1] & = \sum_{i \in \Omega_1} f(X_1 = i) (x_i - E[X_1])^2 \\ & = 0.1(5 - 0.2)^2 + 0.3(1-0.2)^2 + 0.6(-1 - 0.2)^2 \\ & = 3.36
\end{align*}

donde $\Omega_1 = \{-1, 1, 5\}$ es el soporte de la variable $X_1$.\\

De igual forma, se obtiene que 

\begin{align*}
E[X_2] & = 0.6 \\
Var[X_2] & = 2.24
\end{align*}

\begin{align*}
E[X_3] & = 0.4 \\
Var[X_3] & = 3.64
\end{align*}

con $\Omega_2 = \{-1, 1, 3\}$ y $\Omega_3 = \{-1, 0, 6\}$, los soportes de las variables $X_2$ y $X_3$, respectivamente.\\

Para ver la inversión más segura, calculamos la probabilidad de obtener beneficios, es decir, $P(X_i > 0)$ con $i = 1,2,3$. Así, tenemos que\\

\begin{align*}
P(X_1 > 0) & = 1 - P(X_1 \leq 0) \\ & = 1 - P(X_1 = -1) \\ & = 1 - 0.6 \\ & = 0.4
\end{align*}

\begin{align*}
P(X_2 > 0) & = 1 - P(X_2 \leq 0) \\ & = 1 - P(X_2 = -1) \\ & = 1 - 0.4 \\ & = 0.6
\end{align*}

\begin{align*}
P(X_3 > 0) & = 1 - P(X_3 \leq 0) \\ & = 1 - (P(X_3 = -1) + P(X_3 = 0)) \\ & = 1 - 0.7 - 0.2 \\ & = 0.1
\end{align*}

Vemos que la opción más segura para obtener algún beneficio es la segunda, ya que hay 60\% de probabilidades de ganar algo de dinero con esta inversión.\\

Para ver la inversión más arriesgada, calculamos la probabilidad de no perder dinero, es decir, $P(X_i \geq 0)$,

\begin{align*}
P(X_1 \geq 0) & = 1 - P(X_0 < 0) \\ & = 1 - P(X_1 = -1) \\ & = 1 - 0.6 \\ & = 0.4
\end{align*}

\begin{align*}
P(X_2 \geq 0) & = 1 - P(X_2 < 0) \\ & = 1 - P(X_2 = -1) \\ & = 1 - 0.4 \\ & = 0.6
\end{align*}

\begin{align*}
P(X_3 \geq 0) & = 1 - P(X_3 < 0) \\ & = 1 - P(X_3 = -1) \\ & = 1 - 0.2 \\ & = 0.8 
\end{align*}

La opción más arriesgada es la tercera, donde hay más probabilidades de no perder dinero, pero hay un 70\% de no ganar nada, aunque en caso de ganar, ganaremos la mayor cantidad de dinero de todas las inversiones, 6 millones de euros.\\

 

\subsection{Tarea V}



\subsection{Tarea VI}

Si denotamos con $X$ el tiempo (en minutos) hasta que llegan las tartas encargadas a la sede de la SICAV, entonces $X \sim \mathcal{U}(25,45)$.\\

\textcolor{red}{Hacer gráficas}

La esperanza de una de una distribución uniforme de parámetros $(a, b)$ viene dada por

\begin{align*}
E[X] & = \dfrac{a + b}{2}\\
Var[X] & = \dfrac{(b-a)^2}{12}
\end{align*}

Haciendo $a = 25$ y $b = 45$, se tiene que

\begin{align*}
E[X] & = \dfrac{25+45}{2}\\
Var[X] & = \dfrac{(45-25)^2}{12} = \dfrac{100}{3}
\end{align*}

La probabilidad de que $X$ sea, como mucho, 30 minutos es $P(X \leq 30)$.\\

\begin{align*}
P(X \leq 30) & = \int_{25}^{30} \dfrac{1}{45 -25} \ dx \\
& = \dfrac{1}{20}\left[x\right]_{25}^{30} \\ & = \dfrac{1}{4}
\end{align*}

La probabilidad de que $X$ esté comprendido entre 30 y 40 minutos es $P(30 \leq X \leq 40)$.\\

\begin{align*}
P(30 \leq X \leq 40) & = \int_{30}^{40} \dfrac{1}{45 -25} \ dx \\
& = \dfrac{1}{20}\left[x\right]_{40}^{30} \\ & = \dfrac{1}{2}
\end{align*}

Calculamos el percentil 90 de la distribución. Es decir tenemos que encontrar el número $a$ tal que acumula el 90\% de la probabilidad. Es decir,

\begin{align*}
\int_{25}^{a} \dfrac{1}{45 - 25} \ dx = 0.9
\end{align*}

\begin{align*}
\dfrac{1}{20}\int_{25}^{a} dx = \dfrac{1}{20} \left[x\right]_{25}^{a} = \dfrac{1}{20}(a - 25) = 0.9
\end{align*}

Despejando $a$ se tiene que $a = 0.9 \cdot 20 + 25 = 43$.\\

Es decir, el 90\% de las tartas llegará en un tiempo menor o igual a 43 minutos.\\

La probabilidad de que el tiempo $X$ sea mayor de 40 minutos sabiendo que dicho tiempo es de al menos 30 minutos es $P(X > 40 | X > 30)$.

\begin{align*}
P(X > 40 | X > 30) & = \dfrac{P(\{X > 40\}\cap \{X > 30\})}{P(X > 30)} \\
& = \dfrac{P(X > 40)}{P(X > 30)} \\ & = \dfrac{1 - P(X \leq 40)}{1 - P(X \leq 30)} \\
& = \dfrac{1 - \int_{25}^{40} \dfrac{1}{20} \ dx}{1 - \int_{25}^{30} \dfrac{1}{20} \ dx} \\
& = \dfrac{1 - \dfrac{1}{20}(40-25)}{1 - \dfrac{1}{20}(30-25)} \\
& = \dfrac{1- \dfrac{3}{4}}{1 - \dfrac{1}{4}} \\
& = \dfrac{1}{3}
\end{align*}

Por tanto, sabiendo que el tiempo de llegada de las tartas va a ser mayor de 30 minutos, tenemos un 33.\textbar{3} \% de probabilidades de que tarden más de 40 minutos.

\subsection{Tarea VII}

Si ahora denotamos con $X$ al tiempo (en miutos) que dura una llamada cualquiera realizada por un trabajador de VACIS a un teléfono externo, tenemos que $X \sim \mathcal{E}xp(\lambda)$. Se sabe que la duración media de una llamada es de 8 minutos.\\

Así pues, sabemos que $E[X] = \dfrac{1}{\lambda}$. Por otro lado, tenemos que la duración media de las llamadas es de 8 minutos, es decir, $E[X] = 8$. Si igualamos y despajamos $\lambda$ tenemos que $\lambda = \dfrac{1}{8}$. Por tanto, $X \sim \mathcal{E}xp(\dfrac{1}{8})$.\\

\textcolor{red}{Representar las funciones de distribución y de densidad}\\

La esperanza y la varianza de una distribución exponencial con parámetro $\lambda$ viene dada por

\begin{align*}
E[X] & = \dfrac{1}{\lambda} \\
Var[X] & = \dfrac{1}{\lambda^2}
\end{align*}

Sustituyendo $\lambda  = \dfrac{1}{8}$, tenemos que

\begin{align*}
E[X] & = \dfrac{1}{\dfrac{1}{8}} = 8 \\
Var[X] & = \dfrac{1}{\left(\dfrac{1}{8}\right)^2} = 64
\end{align*}

La probabilidad de que una llamada cualquiera a móvil dure menos de 9 minutos es $P(X < 9)$,

\begin{align*}
P(X < 9) & = \int_{0}^{9} \dfrac{1}{8} e^{-\dfrac{x}{8}} \ dx \\
& = -\left[ e^{-\dfrac{x}{8}} \right]_0^9
& = 1 - e^{-\dfrac{9}{8}} \\
& = 0.675
\end{align*}

La probabilidad de que una llamada cualquiera a móvil dure entre 7 y 9 miutos es $P(7 \leq X \leq 9)$,

\begin{align*}
P(7 \leq X \leq 9) & = \int_{7}^{9} \dfrac{1}{8} e^{-\dfrac{x}{8}} \ dx \\
& = -\left[ e^{-\dfrac{x}{8}} \right]_7^9
& = -e^{-\dfrac{9}{8}} + e^{-\dfrac{7}{8}} \\
& = 0.092
\end{align*}

Calculamos el percentil 80 de esta distribución, es decir, el número $a$ tal que la probabilidad acumulada de la distribución es del 80\%. Es decir,

\begin{align*}
\int_{0}^{a} \dfrac{1}{8} e^{-\dfrac{x}{8}} \ dx = 0.8
\end{align*}

\begin{align*}
\int_{0}^{a} \dfrac{1}{8} e^{-\dfrac{x}{8}} \ dx = - \left[ e^{-\dfrac{x}{8}} \right]_0^a \\
& = -e^{\dfrac{a}{8}} + 1 \\
& = 0.8
\end{align*}

Tomando logaritmos a ambos lados de la expresión anterior, se tiene que $a = -8 \cdot \ln 0.2 = 12.875$.\\

Por tanto, el 80\% de las llamadas tienen una duración inferior a 12.875 minutos.\\

Si se realizan un total de 1000 llamadas a móvil, el tiempo total de esas llamadas es de 8000 minutos, ya que la duración media de una llamada cualquiera es de 8 minutos. Así 1000 llamadas $\cdot$ 8 minutos/llamada = 8000 minutos. Si queremos ser más precisos, con nuestra estimación podemos calcular un intervalo de probabilidad al 90\%.\\

Normalmente, calculamos los intervalos de probabilidad bilaterales, pero en este caso, por tratarse de una distribución con alta simetría hacia la derecha, calcularemos el intervalo con los cuantiles 0 y 0.90. Si calculamos estas probabilidades como en el apartado anterior tenemos que el intervalo es $(0, 18.42)$. Es decir, el 90\% de las llamadas estarán entre 0 y 18.42 minutos. Por tanto, el número de minutos a diario estará entre 0 y 18420 minutos diarios.\\

\textcolor{red}{Hacer gráfica con el intervalo de probabilidad} 

\section{Conclusiones}
   

\newpage

\section{Código R}

\verbatiminput{../caso_ii2.0.R}


\end{document} 